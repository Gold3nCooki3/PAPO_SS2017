\section{Parallelisierungsschema}

Das Parallelisierungschema von SMITH ergibt sich durch die Datenaufteilung.\\
Das Kaufhaus wird als dreidimensionale Matrix betrachtet, die zeilenweise auf die vorhandenen MPI-Prozesse aufgeteilt wird. Diese Daten bleiben statisch und werden nicht verändert, bis auf die Menge der Artikel in einem Regal.\\
Die Aufteilung der Kunden ist abhängig von ihren Einkaufslisten. So werden die Kunden zwar innerhalb bestimmter Prozesse in das Kaufhaus eingesetzt (auf den Eingangs- und Ausgangsfeldern), jedoch verteilen diese sich um ihre zufallsgenerierte Einkaufsliste zu bearbeiten. Wird hierbei eine Zeile betreten, welche zu einem anderen MPI-Prozess gehört, so wird die Kunden-Entität mit allen anderen auszutauschenden Entitäten gesammelt und in gebündelt einer zum nächsten Prozess verschickt. Dieser nächste Prozess kann entweder einen geringeren oder höheren Zeilenindex haben, sich aber im selben "`Stockwerk"' befinden wie derjenige, der sendende, oder aber über oder unterhalb des Prozesses liegen. Hierbei kann es ein Maximum von sechs Nachbarprozessen geben, bedingt durch die zeilenweise Aufteilung.