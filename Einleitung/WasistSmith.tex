\section{Über das Projekt}
SMITH ist eines der Ergebnisse des Praktikums \enquote{Parallele Programmierung} im Sommersemester 2017. Löst man das Akronym auf, so ergibt sich der Name \enquote{Simulate Malls in the Hood}, der schon verrät, wovon das Projekt handelt: Vom Simulieren von Kaufhäusern.

Aufgabenstellung war es ein Programm zu schreiben, welches mit MPI parallelisiert wurde und auf mehreren Rechenknoten lauffähig ist.
Dies erreicht SMITH, indem er eine große Matrix als Eingabe erhält, diese auf verschiedene Prozesse aufteilt und anschließend Kunden und Mitarbeiter in dieser Matrix, die ein Kaufhaus darstellt, erzeugt und deren Aktivitätenliste abarbeitet. Die Parallelisierung findet also in der Verteilung der Daten statt, Kunden und Mitarbeiter werden anhand Ihrer Position im Kaufhaus den Rechenknoten zugeteilt.

Weiterhin besteht SMITH aus einem Zusatzprogrammen zur Erstellung von Kaufhäusern und zur Generierung von Kunden und Einkaufslisten.
Gedacht ist SMITH als Benchmark. Verschiedene Eingaben können auf Systemen verglichen werden und die Skalierung von Systemen kann beobachtet werden. Erstellt man ein großes Kaufhaus und verwendet wenig Prozesse, so nimmt der A*-Algorithmus einen Großteil der Rechenzeit in Anspruch, es wird also hauptsächlich die Rechenhardware getestet. Verwendet man jedoch ein relativ kleines, einfach aufgebautes Kaufhaus und nutzt viele Prozesse zur Ausführung, so nimmt MPI einen immer größeren Anteil an der Ausführungszeit ein. Mit der Möglichkeit Kunden und Einkaufslisten zufällig zu erstellen können weiterhin Tests über längere Laufzeit ausgeführt werden und die durchschnittliche Leitung von Systemen kann verglichen werden.